\documentclass{article}\usepackage[]{graphicx}\usepackage[]{xcolor}
% maxwidth is the original width if it is less than linewidth
% otherwise use linewidth (to make sure the graphics do not exceed the margin)
\makeatletter
\def\maxwidth{ %
  \ifdim\Gin@nat@width>\linewidth
    \linewidth
  \else
    \Gin@nat@width
  \fi
}
\makeatother

\definecolor{fgcolor}{rgb}{0.345, 0.345, 0.345}
\newcommand{\hlnum}[1]{\textcolor[rgb]{0.686,0.059,0.569}{#1}}%
\newcommand{\hlsng}[1]{\textcolor[rgb]{0.192,0.494,0.8}{#1}}%
\newcommand{\hlcom}[1]{\textcolor[rgb]{0.678,0.584,0.686}{\textit{#1}}}%
\newcommand{\hlopt}[1]{\textcolor[rgb]{0,0,0}{#1}}%
\newcommand{\hldef}[1]{\textcolor[rgb]{0.345,0.345,0.345}{#1}}%
\newcommand{\hlkwa}[1]{\textcolor[rgb]{0.161,0.373,0.58}{\textbf{#1}}}%
\newcommand{\hlkwb}[1]{\textcolor[rgb]{0.69,0.353,0.396}{#1}}%
\newcommand{\hlkwc}[1]{\textcolor[rgb]{0.333,0.667,0.333}{#1}}%
\newcommand{\hlkwd}[1]{\textcolor[rgb]{0.737,0.353,0.396}{\textbf{#1}}}%
\let\hlipl\hlkwb

\usepackage{framed}
\makeatletter
\newenvironment{kframe}{%
 \def\at@end@of@kframe{}%
 \ifinner\ifhmode%
  \def\at@end@of@kframe{\end{minipage}}%
  \begin{minipage}{\columnwidth}%
 \fi\fi%
 \def\FrameCommand##1{\hskip\@totalleftmargin \hskip-\fboxsep
 \colorbox{shadecolor}{##1}\hskip-\fboxsep
     % There is no \\@totalrightmargin, so:
     \hskip-\linewidth \hskip-\@totalleftmargin \hskip\columnwidth}%
 \MakeFramed {\advance\hsize-\width
   \@totalleftmargin\z@ \linewidth\hsize
   \@setminipage}}%
 {\par\unskip\endMakeFramed%
 \at@end@of@kframe}
\makeatother

\definecolor{shadecolor}{rgb}{.97, .97, .97}
\definecolor{messagecolor}{rgb}{0, 0, 0}
\definecolor{warningcolor}{rgb}{1, 0, 1}
\definecolor{errorcolor}{rgb}{1, 0, 0}
\newenvironment{knitrout}{}{} % an empty environment to be redefined in TeX

\usepackage{alltt}
\usepackage{natbib} %for the bibliography
\setlength{\bibsep}{0pt plus 0.3ex}
\bibliographystyle{apalike} %sets bibliography to apa citation
\usepackage[margin = 0.5in]{geometry}
\usepackage{hyperref}
\hypersetup{colorlinks = true,citecolor=black}
\usepackage{float}
\usepackage{multicol}
\IfFileExists{upquote.sty}{\usepackage{upquote}}{}
\begin{document}

\vspace{-1in}

\title{Lab 2 Report -- Math240 -- Compuational Statitsics}
\author{
  Jeremy Artiga \\
  No Affiliation \\
  No Department \\
  {\tt{jartiga@colgate.edu}}
}

\date{}

\maketitle

\begin{multicols}{2}
\begin{abstract}
In this lab, we will examine a set of non-copyrighted \texttt{.WAV} files
from a Music directory using Base \texttt{R} commands and packages. More specifically,
we will analyze data within the \texttt{.WAV} files to build a batch file of 
commands that will allow data processing within the open-source program Essentia,
which allows for music analysis, description, and synthesis. Our batch file will
allow us to process \emph{large} sets of \texttt{.WAV} files within Essentia within
seconds w/o the need to process each file manually. We will also experiment with the
\texttt{jsonlite} package, which allows us to extract musical analysis data from
given .JSON files.
\end{abstract}

\textbf{Keywords:} This lab will enable us to practice (1) installing, loading, and learning
about libraries/packages; (2) working with character objects; (3) writing \texttt{for()} loops;
and (4) creating/accessing elements within vectors and lists

\section{Introduction}
For this lab, we will:

\begin{enumerate}
  \item Install the \texttt{stringr} package for \texttt{R} \citep{stringr}. 
  \item Figure out the commands needed to access a directories and \texttt{.WAV} files 
  within a Music directory. 
  \item Create character/vector/list objects to store data within a Music
  directory.
  \item Utilize \texttt{for()} loops to iterate through vector/lists
  to search/mutate data in a Music directory. 
  \item Utilize functions in the \texttt{stringr} package to subset
  and track \texttt{.WAV} file data. 
  \item Write to a \texttt{.txt} file using \texttt{writeLines()}. 
  \item Install and load the \texttt{jsonlite} package for \texttt{R}.
  \item Utilize the \texttt{jsonlite} package to load musical analysis
  data. \citep{jsonlite}
  \item Utilize objects to save necessary musical data.
\end{enumerate}

\columnbreak

\section{Methodology}
In this section we will go through the process of loading the needed directory/files and 
packages; creating objects; and utilizing coding structures to accomplish our objectives. \\

We ultimately want to write command lines in the following format: 
\begin{center}
\begin{footnotesize}
  \texttt{streaming}\verb|_|\texttt{extractor}\verb|_|\texttt{music.exe "EXAMPLE.wav" "EXAMPLE.json"}
\end{footnotesize}
\end{center}

\subsection{Package and Object Set Up}
We will explore a file directory called MUSIC, which contains all the \texttt{.WAV} files we will
examine, this directory will be downloaded and saved into the same directory as our \texttt{R} script.

We will then begin our R script by loading the \texttt{stringr} package using \texttt{library()}, this
will allows us to perform string mutations. Now, we setup the objects which will store the subsets of our
directory (i.e. albums, song lists, etc.) using descriptive variable naming. All the objects will correspond
to data we need to store in order to write our Essentia command lines.

\subsection{Code Structures}

\subsubsection{Subsetting File Data}
In order to write our batch file command lines, we will need to go through every \texttt{.WAV} file
in the sub directories of the MUSIC directory, these sub directories are the albums of our
\texttt{.WAV} files (i.e. song files). \\

We will begin by utilizing a \texttt{for()} loop to gather all the paths of the albums sub directories within the MUSIC directory. In other words, we are getting all the albums stored in the MUSIC directory. \\

We then have to go through each album sub directory to obtain all the \texttt{.WAV} files in the directory.
This will require the use of nested \texttt{for()} loops, one for the album directories and one for the
\texttt{.WAV} files. This will give us all of the \texttt{.WAV} files. \\

Next we will have to gather all the \texttt{.WAV} files and and all our album directories and construct full 
directory paths with the albums and songs we obtained earlier. We will utilize another nested \texttt{for()} 
loop for this task as well. I also added in a structure that filters the \texttt{.WAV} files from the album
directory.

\pagebreak

\subsubsection{Constructing Batch File Commands}
Now we have to grab only the artists names and track names using the \texttt{stringr} package. By utilizing
the naming convention for files, we can subset the track and artists names using \texttt{str}\verb|_|\texttt{sub()}
and \texttt{str}\verb|_|\texttt{split()}, we'll perform this operation on all the \texttt{.WAV} files using a \texttt{for()} loop. \\

We will construct the \texttt{.json} output files by pasting the artists names, album name, and track name
together with a \texttt{.json} extension at the end using the \texttt{paste()} function.
The following format is as follows:

\begin{center}
\begin{footnotesize}
\texttt{[artist name]-[album name]-[track name].json}
\end{footnotesize}
\end{center}

Finally, we will paste together the string \\
"streaming\verb|_|extractor\verb|_|music.exe" with our \texttt{.WAV} file
name and our \texttt{.json} file name to produce the batch file command lines.

Using the \texttt{writeLines()} we can write our batch file commands into a TXT file called batfile.txt,
which will contain all the music processing command lines for Essentia to execute.

\subsection{Extracting Data Using the \texttt{jsonlite} Package}
Suppose we have a .JSON output song file we want to analyze. We can use the \texttt{jsonlite} package to obtain
all sorts of musical data. From the key, tone, all the way to danceability!
To illustrate this, we will will use the .JSON output for "Au Revoir (Adios) on the Talon Of The Hawk" album by The Front Bottoms. \\

We will firstly load the \texttt{jsonlite} package using the \texttt{library()} function. Then we want to use
\texttt{str}\verb|_|\texttt{split()} to extract the artists, album, and track names form the file. This will
be useful for a later project.  \\

Now we want to load the JSON file into \texttt{R} using the \texttt{fromJSON()} function. This will load a list
of all the musical data attributed to our file.
Finally, we want to extract the data in the resulting list and store the into objects. We can do this by
accessing lower-level elements in the list (much like a directory) using \verb|$| syntax.

\columnbreak

\section{Discussion}
By utilizing packages and base \texttt{R} coding structures such as \texttt{for()} loops, objects, and accessing
elements within vectors/lists, we were able to create an Essentia command batch file from examination of 
\texttt{.WAV} files in the music directory. This not only enabled us to practice our abilities to utilize the said
coding structures, but it also allowed us to build an \texttt{R} program which could analyze any sort of directory
and pull the \texttt{.WAV} files for the sake of data processing into Essentia. We also figured out how to extract
data from .JSON files using the \texttt{jsonlite} package to gather information on specific albums/songs, which we could potential utilize to analyze songs and make general insights based on this analysis in another project.

\begin{footnotesize}
\bibliography{bib}
\end{footnotesize}






 
\end{multicols}{2}
\end{document}
